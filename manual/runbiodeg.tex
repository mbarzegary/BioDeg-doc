After installing/compiling \biodeg{} as described in Section~\ref{sec:installation}, we are ready to run it. There are 2 ways to run \biodeg{} simulations: 1) using the UI to configure and execute \biodeg{}, or 2) running \biodeg{} directly from command line and provide simulation parameters via command line arguments. Moreover, in a hybrid approach, the UI can be used to configure and generate the command for method \#2, which can be useful in which you want to configure the simulation only and run it later in another environment like on a super-computer or cluster.

The UI can be run simply by double clicking on the \verb|BioDeg-UI.exe| in Windows or by executing \verb|./BioDeg-UI| in Linux. For running \biodeg{} directly, one need to execute the following command:
\begin{verbatim}
$ mpirun -n N FreeFem++-mpi BioDeg-core/src/main.edp <command line args>
\end{verbatim}
in which N defines the number of MPI processes to be used. The full list of command line arguments can be found in Section \hyperref[sec:index]{Index}.

In order to clarify and demonstrate the procedure of performing simulations using \biodeg{}, 2 step-by-step examples are provided in Section~\ref{sec:config}, showing how to configure and run simulations with and without the UI. The mesh files needed to run these examples can be found in the \verb|examples| directory.

\subsection{Configuring the simulation} \label{sec:config}


\subsubsection{Example 1}\label{sec:example1}

%Let us consider the first example given in the folder
%\verb|/dftfe/demo/ex1|, where we compute the 

\subsubsection{Example 2}\label{sec:example2}
