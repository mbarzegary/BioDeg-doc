All the underlying installation instructions assume a Linux operating system. We assume standard tools and libraries like CMake, compilers- (C, C++ and Fortran), and MPI libraries are pre-installed. Most high-performance computers would have the latest version of these libraries in the default environment. 

\subsection{Compiling and installing external libraries}

\subsubsection{Instructions for PETSc and Qt}

\subsubsection{Instructions for FreeFEM}

\subsection{Obtaining and Compiling \biodeg{}}

Assuming that you have already installed the above external dependencies, next follow the steps below to obtain and compile \biodeg{}.

%\begin{enumerate}
%\item Obtain the source code of the current release of \dftfe{} with all current patches using \href{https://git-scm.com/}{Git}:
%\begin{verbatim}
%$ git clone -b release https://github.com/dftfeDevelopers/dftfe.git
%$ cd dftfe
%\end{verbatim}
%Do \verb|git pull| in the \verb|dftfe| directory any time to obtain new patches that have been added since your \verb|git clone| or last \verb|git pull|.
%If you are not familiar with Git, you may download the current release tarball from the \href{https://sites.google.com/umich.edu/dftfe/download}{Downloads} page in our website, but downloading via Git is recommended to avail new patches seamlessly. 
%\end{enumerate}
